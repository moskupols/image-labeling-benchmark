\RequirePackage{amsmath}
\documentclass[hidelinks]{llncs}

\usepackage[russian,english]{babel}

\usepackage{amssymb}

\usepackage{cmap}
\usepackage[utf8]{inputenc}
\usepackage[T2A]{fontenc}

\usepackage{tikz}

\usepackage[font=small]{caption}
\usepackage{subcaption}

\usepackage{hyperref}
\usepackage{cleveref}

\usepackage{listings}
\usepackage{xcolor}

\usepackage{xspace}

\usepackage{tabularx}
\usepackage{graphicx}


\newcommand{\texten}[1]{#1}
\newcommand{\textru}[1]{}

%% Чтобы переключиться на русский язык в некоторых секциях, можно 
%% раскомментировать следующие две команды
\renewcommand{\texten}[1]{}
\renewcommand{\textru}[1]{\foreignlanguage{russian}{\emph{#1}}{}}

%% Чтобы оставить оба языка, можно раскомментировать
%% следующие две команды
% \renewcommand{\texten}[1]{#1}
% \renewcommand{\textru}[1]{\\ \foreignlanguage{russian}{\emph{#1}}}


\tikzset{every picture/.append style={scale=0.5}}

\tikzstyle{cell}=[rectangle,draw=gray,dashed,semithick, minimum size=1.05cm]
\tikzstyle{graybox}=[rectangle,thick, minimum size=0.4cm, fill=lightgray]
\tikzstyle{whitebox}=[rectangle,draw=black,thick, minimum size=0.5cm]

% \captionsetup[subfigure]{subrefformat=simple,labelformat=simple}
% \renewcommand\thesubfigure{(\alph{subfigure})}

\lstset{
  language=c++,
  basicstyle=\ttfamily \scriptsize,
  lineskip={-1.5pt},
  columns=fixed,
  basewidth=0.5em,
  keywordstyle=\bfseries\color{blue!60!black},
  commentstyle=\itshape\color{gray!80!black},
  identifierstyle=\color{violet!60!black},
  stringstyle=\color{orange},
  frame=single
  % numbers=left
}

\newcommand{\norm}[1]{\left\lVert#1\right\rVert}
\newcommand*{\N}{\mathbb{N} \xspace}
\DeclareMathOperator*{\argmin}{arg\,min}

\renewcommand{\C}{\texttt{C} \xspace}
\newcommand{\CXX}{\texttt{C++} \xspace}

\begin{document}

\title{Linear Variation and an Optimization of Algorithms for Connected
Components Labeling in a Binary Image}

\author{Fedor Alekseev\inst{1} \and Mikhail Alekseev\inst{2}\inst{3}
\and Artyom Makovetskii\inst{2}\inst{4}}

\institute{Moscow Institute of Physics and Technology State University, Dolgoprudny, Russia\\
\email{alekseev@phystech.edu}
\and
Chelyabinsk State University, Chelyabinsk, Russia
\and
\email{alexeev@csu.ru}
\and
\email{artemmac@mail.ru}}

\maketitle              % typeset the title of the contribution

\begin{abstract}
The linear variation is a topological characteristic of the function of two variables.
The problem of linear variation computing can be reduced to the problem of counting connected components in a binary image with eight-connected connectivity.
%The new hybrid method for the problem is presented.
%The method is essentially a modification of a two-scan algorithm for labeling
%that groups the pixels into $2 \times 2$ cells.
This is usually done through connected components labeling, and many approaches for that are known.
This paper presents an efficient way to convert the initial binary image with 8-connectivity into a condensed (non-binary) image with new connectivity that is 2 times smaller in each dimension than the initial image.
Most approaches for connected components labeling are still valid for the new image.
As time and memory consumptions of conventional approaches usually depend linearly on the number of pixels,
running them on the condensed image can be up to 4 times more efficient than running on the initial image.
The method is essentially a modification of some known raster algorithms for labeling that groups pixels into $2 \times 2$ cells.
A performance benchmark of the proposed  method applied to some approaches on noise images is provided.

\keywords{Binary raster image, connected component labeling, pattern recognition}
\end{abstract}

\section{Introduction}

% Makovetskii part

In image restoration it is often necessary to consider the following problem.
How to restore an original undistorted image $v$, if it is known an observed
image $u_0$ and the relation between $u_0$ and $v$:
\begin{equation}
  u_0=v+n,
  \label{eq:imageRelation}
\end{equation}
where $n$ is a noise?
A common way for solving the problems \eqref{eq:imageRelation} is to use the
variational functionals.
One of the most widely used approaches is total variation
regularization\cite{mak1}.
Let $J(u)$ be the following functional:
\begin{equation}
  J(u) = \frac12 \norm{u - u_0}^2 + \lambda TV(u),
  \label{eq:jFunctional}
\end{equation}
where $\norm{\cdot}$ is the $L_2$ norm and all function we consider belong to the class
$BV(\Omega)$, i.\,e. the class of bounded variation on the set $\Omega$ functions.
%
The expression $\frac12 \norm{n-n_0}^2$ is called a fidelity term and
$\lambda TV(u)$ is called a regularization term and $\lambda$ is regularization
parameter.

\begin{equation}
  TV(u) = \int\limits_{\Omega} |\nabla u|\,dx\,dy.
  \label{eq:TV}
\end{equation}

Let $u_*$ be extremal function for the following variational problem:
\begin{equation}
  u_* = \argmin_{u \in BV(\Omega)} J(u).
  \label{eq:ustar}
\end{equation}
The \autoref{eq:ustar} is called total variation regularization problem.

The norm and modulus of the gradient are metrical characteristic of a function of
two variables.
Continuous functions of two variables also have a set of topological
characteristics referred to as linear variations.
Kronrod\cite{mak2}
introduced the notion of a regular component of the level set of a
continuous function of two variables.
The simplest topological characteristic in the linear variation theory is a number
of regular components for all level sets of a function.
Full information about linear variations of a function is contained in the
one-dimensional tree of a function of two variables.
In \cite{mak3,mak4,mak5}
was discussed the necessity of the using a linear variation in the
image restoration theory.

Let $\Phi_u(t)$ be the number of regular components of a level set $t$
for a continuous function.
The first Kronrod's linear variation is defined as
\begin{equation}
  V(u) = \int\limits_{-\infty}^{+\infty} \Phi_u(t)\,dt.
  \label{eq:KronrodVariation}
\end{equation}

Let $w$ be a binary discrete function $w=(w_{i,j})$, where $w_{i,j} \in \{0,1\}$,
for all pairs $(i,j)$.
A subset of such pairs $(i, j)$ when $w_{i,j}=1$ and all elements of the subset are
connected by the 8-connectivity, is called the connected component of the binary
function $w$.
For a number $k \in \N$ and a discrete function $u$ we define the following
indicator function $\chi$:
\begin{equation}
  \chi_k (u_{i,j}) =
    \begin{cases}
      1, & u_{i,j} \ge k \\
      0, & u_{i,j} < k
    \end{cases}.
  \label{eq:chiIndicator}
\end{equation}

\begin{definition}
  The number $V_k(u_{i,j})$ of connected components for a level $k$, $k \in \N$ of
  the discrete function $u$ is called the number of connected components of the 
  binary discrete function $\chi_k(u_{i,j})$.
\end{definition}

\begin{definition}
  The linear variation $V(u_{i,j})$ of a discrete function  is defined as follows:
  \begin{equation}
    V(u_{i,j}) = \sum_{k=0}^{+\infty} V_k(u_{i,j})
    \label{eq:V}
  \end{equation}
\end{definition}

Let us compute the discrete gradient $\nabla u_{i,j}$ of $u$ at $(i,j)$ as
\begin{equation}
  \nabla u_{i,j} = (u_{i+1,j} - u_{i,j}, u_{i,j+1} - u_{i,j}).
  \label{eq:gradient}
\end{equation}

Suppose that if the pair $(i,j)$ is outside of the domain of the function $u$, then 
$u_{i,j} = 0$.
The main problem of the computer implementation of the linear variation is 
computation of $V_k(u_{i,j})$ for given k.

% Alekseev part

This problem is equivalent to the problem of counting the number of connected
components (CC) in a binary image with 8-connectivity.
This is usually done through CC labeling, and many approaches for that are
known\cite{hechao}.

This paper presents an efficient way to convert the initial binary image
with 8-connectivity into
a condensed (non-binary) image with new connectivity that is $2$ times smaller in
each dimension than the initial image.
Most approaches for CC labeling are still valid for the new image.
As time and memory consumptions of conventional approaches usually depend linearly on
the number of pixels, running them on the condensed image can be
more efficient than running on the initial image.

It is worth noting that our method exploits an important property of 8-connectivity
that does not hold for 4-connectivity, so it is not applicable for the latter case.

\section{$2\times2$ condensation}

\texten{
We will use the fact that for the case of 8-connectivity all $4$ pixels of any
$2 \times 2$ square are adjacent to each other, and so are contained in the same CC
and should have same labels upon completion of the labeling algorithm.
So instead of labeling single pixels we can label $2 \times 2$ groups of pixels,
or ``cells''.
}\textru{
Наблюдение: с 8-связностью в любом квадратике $2 \times 2$ любые два пикселя
находятся рядом. Значит если это пара объектных пикселей, то они находятся в одной
КС, и значит каждый такой квадратик пересекается не более чем с одной КС.
Поэтому вместо маркировки отдельных пикселей можно маркировать такие квадраты.
Назовём их клетками.
}


\texten{
Consider a $2N \times 2M$ binary image\footnote{
  For convenience in this paper we consider only input images with even size
  in both dimensions. If this is not the case, one can easily append a row or
  a column of background pixels to the image.
}
specified by a binary predicate $b(x, y)$,
returning $1$ in case the pixel at the given coordinates is black (object pixel),
and $0$ otherwise (background pixel).
For convenience we use 0-indexation throughout the paper, so the topmost row has
row index $y=0$, and the leftmost column has column index $x=0$.
}\textru{
Рассмотрим бинарное изображение размера $2N \times 2M$\footnote{
  Для удобства мы рассматриваем только изображения с чётной высотой и шириной.
  В общем случае при необходимости можно добавить строку или столбец с фоновыми
  пикселями.
},
заданное предикатом $b(x, y)$, который возвращает $1$, если переданы координаты
объектного пикселя, и $0$ иначе.
Мы везде будем использовать $0$-индексацию: верхняя строка и левый столбец имеют
номер $0$.
}


\texten{
The condensed image will be of size $N \times M$.
Each pixel of the condensed image will unambiguously represent the configuration
of the corresponding $2\times 2$ cell of pixels of the initial binary image.
So the condensed image will contain pixels of $2^4 = 16$ colors.
We suggest the function $c(x, y)$ denoting the color of a pixel of condensed image
to be the following:
}\textru{
Сжатое изображение будет иметь размер $N \times M$.
Каждый пиксель полученного изображения должен будет однозначно задавать конфигурацию
соответствующих четырёх пикселей исходного изображения, поэтому он может быть одного
из $2^4 = 16$ цветов. Предлагается такая функция $c(x,y)$, кодирующая четыре исходных
пикселя в цвет клетки:
}
\begin{align*}
  c(x, y) &= 2^0 b(2x, 2y)   + 2^1 b(2x+1, 2y) \\
          &+ {} 2^2 b(2x, 2y+1) + 2^3 b(2x+1,2y+1)
\end{align*}

\texten{
So the value of each pixel of the initial image is stored in the corresponding
bit in color of the corresponding pixel of the condensed image.
Note that if $c(x, y) = 0$ for some $(x, y)$, then there are no object pixels in
this cell, and no label should be assigned to this whole cell. The enumeration of
pixels inside one cell is shown on \autoref{fig:pixelsEnumeration}.
}\textru{
Таким образом значение каждого из исходных пикселей просто записано в соответствующий
бит в итоговом пикселе. Важно отметить, что если все четыре исходных пикселя были
фоновые, то и клетка будет иметь цвет 0. Нумерация пикселей внутри клетки показана на
\autoref{fig:pixelsEnumeration}.
}

\begin{figure}[t]
  \centering
  \begin{tikzpicture}
    \node[cell] at (0.5, 0.5) {};
    \foreach \x/\y [count=\i from 0] in {0/1,1/1,0/0,1/0} {
      \node[whitebox] at (\x,\y) {\i};
    }
  \end{tikzpicture}
  \caption{Cell pixels enumeration}
  \label{fig:pixelsEnumeration}
\end{figure}

\texten{
We need also to define connectivity function for the condensed image.
We cannot actually just use 8-connectivity, as whether two cells are adjacent
cells of one CC depends not only on their coordinates, but also on their
configuration. See \autoref{fig:connectivity:examples} for examples.
}\textru{
Ещё нужно дать определение связности для сжатого изображения. Мы уже не можем
использовать 8-связность: теперь нельзя сказать, связаны ли две клетки, не
проанализировав их цвета. Смотри \autoref{fig:connectivity:examples}.
}

\begin{figure}
  \centering
  \begin{subfigure}[t]{0.3\linewidth}
    \centering
    \begin{tikzpicture}
      \foreach \x in {0.5,2.5}
        \node[cell] at (\x,0.5){};
      \foreach \x/\y in {0/0, 1/1, 3/0, 3/1}
        \node[whitebox] at (\x, \y){};
    \end{tikzpicture}
    \caption{Nearby cells that are not directly connected.}
  \end{subfigure}
  \quad
  \begin{subfigure}[t]{0.3\linewidth}
    \centering
    \begin{tikzpicture}
      \foreach \x in {0.5,2.5}
        \node[cell] at (\x,0.5){};
      \foreach \x/\y in {0/0, 1/1, 2/0, 3/1}
        \node[whitebox] at (\x, \y){};
    \end{tikzpicture}
    \caption{Directly connected cells.}
  \end{subfigure}
  \quad
  \begin{subfigure}[t]{0.3\linewidth}
    \centering
    \begin{tikzpicture}
      \foreach \x in {0.5,4.5}
        \node[cell] at (\x,0.5){};
      \foreach \x/\y in {0/0, 0/1, 1/1, 4/0, 4/1, 5/0}
        \node[whitebox] at (\x, \y){};
    \end{tikzpicture}
    \caption{Cells that are not nearby in terms of 8-connectivity, cannot be
    directly connected.}
  \end{subfigure}
  \caption{Cells connectivity examples}
  \label{fig:connectivity:examples}.
\end{figure}

\texten{
However, for every variant of relative placement of two nearby cells there is
a mask of pixels of each cells that are important for direct connectivity of
this pair of cells.
So two cells are considered directly connected, if they are nearby
in terms of 8-connectivity and each of them contains at least one object pixel
in the mask of important pixels corresponding to relative placement of the cells.
The masks are shown in \autoref{fig:connectivity:masks}.
}\textru{
Тем не менее, можно заметить, что анализировать можно не все биты цвета, а только
некоторую маску тех бит, которые важны для связности двух конкретных клеток.
Эти маски показаны на \autoref{fig:connectivity:masks}.
}

\begin{figure}
  \centering
  \begin{subfigure}[t]{0.2\linewidth}
    \centering
    \begin{tikzpicture}
      \node[cell] at (0.5, 2.5) {};
      \node[cell] at (2.5, 0.5) {};
      \node[graybox] at (1, 2) {3};
      \node[graybox] at (2, 1) {0};
    \end{tikzpicture}
  \end{subfigure}
  \begin{subfigure}[t]{0.2\linewidth}
    \centering
    \begin{tikzpicture}
      \node[cell] at (0.5, 2.5) {};
      \node[cell] at (0.5, 0.5) {};
      \node[graybox] at (0, 2) {2};
      \node[graybox] at (1, 2) {3};
      \node[graybox] at (0, 1) {0};
      \node[graybox] at (1, 1) {1};
    \end{tikzpicture}
  \end{subfigure}
  \begin{subfigure}[t]{0.2\linewidth}
    \centering
    \begin{tikzpicture}
      \node[cell] at (0.5, 0.5) {};
      \node[cell] at (2.5, 2.5) {};
      \node[graybox] at (1, 1) {1};
      \node[graybox] at (2, 2) {2};
    \end{tikzpicture}
  \end{subfigure}
  \begin{subfigure}[t]{0.25\linewidth}
    \centering
    \begin{tikzpicture}
      \node[cell] at (0.5, 0.5) {};
      \node[cell] at (2.5, 0.5) {};
      \node[graybox] at (1, 0) {3};
      \node[graybox] at (1, 1) {1};
      \node[graybox] at (2, 0) {2};
      \node[graybox] at (2, 1) {0};
    \end{tikzpicture}
  \end{subfigure}
  \caption{Masks of important pixels for all four cases of relative placement
  of two nearby cells. Those pixels are filled in gray, with their numbers within
  cell specified.}
  \label{fig:connectivity:masks}
\end{figure}

\texten{
The idea is that if we naturally encode mask pixels as 4-bit numbers in the
way similar to how we defined $c$, we can express the predicate denoting whether
two nearby cells
are directly connected efficiently with just two bitwise and one logical AND
operations.
As an example, \autoref{fig:connectivity:code} compares possible codes in \C checking
if the labels of the current pixel or cell and the one right on top of it
should be same for classic 8-connectivity and for the new connectivity.
Note that as bitwise operations are usually relatively cheap, we added only a
little complication compared to decreasing the number of pixels to process by the 
factor of 4.
}\textru{
Идея в том, что с нашим естественным способом кодирования конфигурации клеток в
4-разрядных числах, предикат, определяющий связность двух соседных клеток,
может быть представлен всего двумя битовыми и одной логической операцией И.
В качестве примера, на \autoref{fig:connectivity:code} приведено сравнение
возможных фрагментов программы на языке \C\hspace{-0.8em}, проверяющих, связан
ли текущий пиксель (или клетка) с пикселем (или клеткой) над ним (ней).
Стоит отметить, что хотя конструкция
усложнилась, битовые операции как правило очень эффективны, и это усложнение, скорее
всего, незначительно по сравнению с уменьшением количества рассматриваемых позиций в 4
раза.
}

\begin{figure}
  \centering
  \begin{subfigure}[t]{0.475\linewidth}
    \centering
    \begin{lstlisting}
    if (b(x, y)) {
      if (b(x, y-1)) {
        // copy labels, or union label sets
      }
      // consider other directions
    }
    \end{lstlisting}
    \caption{Binary image with 8-connectivity.}
  \end{subfigure}
  \quad
  \begin{subfigure}[t]{0.475\linewidth}
    \centering
    \begin{lstlisting}
    if (c(x, y)) {
      if ((c(x, y-1) & ((1<<2) | (1<<3)))
       && (c(x, y)   & ((1<<0) | (1<<1)))) {
        // copy labels, or union label sets
      }
      // consider other directions
    }
    \end{lstlisting}
    \caption{Condensed image with new connectivity.}
  \end{subfigure}
  \caption{Checking the nearby top pixel/cell.}
  \label{fig:connectivity:code}
\end{figure}

\section{Benchmarks}

In this section we present performance benchmarks of some popular methods for
counting CC both with and without the $2 \times 2$ cells optimization.

\subsection{Algorithms overview}

\texten{
As each of these methods, omitting DTSUF could be used with both condensed and
non-condensed images, for convenience we denote height and width of the image
on input of each algorithm as $H$ and $W$.
}\textru{
Так как все рассматриваемые методы, кроме DTSUF, можно применять как для сжатого,
так и для несжатого изображения, мы обозначим размеры входного изображения
каждого изображения за $H$ и $W$.
}

\texten{
Also to avoid repeating ``pixel or cell'' we will just use the word ``pixel''
meaning the appropriate raster for non-condensed and condensed images.
When we mention neighbors of some pixel, we mean pixels that are directly connected
to it in terms of appropriate connectivity.
}\textru{
По аналогии, чтобы избежать повторения ``пиксель или клетка'' мы будет просто
использовать слово ``пиксель'', имея в виду растр входного изображения.
Под соседями пикселя будем понимать пиксели, прямо связанные с ним в терминах
соответствующей связности.
}

\subsubsection{DFS}

The first algorithm we chose for the benchmark is Depth First Search graph traversal.
Graph traversal algorithms are the general approach for CC labeling in graphs, and
DFS is arguably the most simple.
Although not intended for image labeling and
suboptimal in most cases, it still has linear in the number of pixels worst-case
complexity in both time and memory.

Like most approaches, the algorithm has an outer loop over all pixels of the image.
Each time an object pixel (non-empty cell, in case it is actually a cell)
which is not labeled yet is encountered, DFS traversal is launched from that pixel.
The traversal labels all the object pixels of the same CC, launching recursively
from all unlabeled neighbors of current pixel.

\subsubsection{SUF}

\texten{
Another general approach for graph CC labeling which is more popular in image
labeling is using some kind of a data structure to maintain disjoint sets of pixels
that are considered to be in the same CC.
}\textru{
Другой, более популярный в этом контексте общий подход к подсчёту компонент
связности состоит в том, чтобы использовать структуру данных для поддержания
набора непересекающихся множеств пикселей, чем, по сути, КС и являются.
}

\texten{
The structure is commonly called Union-Find, as the two essential operations
it is supposed to support are Union, which is used to merge some two of the
disjoint sets in one, and Find, which is used to find the representative element
of the set some element belongs to, so that we can always tell if two elements are
in the same set.
The structure can be implemented so that the amortized time complexity for each
operation is $O(\alpha(HW))$, where $\alpha$ is the inverse Ackermann function,
and $\alpha(HW) < 5$ for all remotely practical image sizes\cite{CLRS}. % TODO links
}\textru{
Эту структуру принято называть Union-Find, так как две основные операции,
которые она поддерживает, это Union~--- объединение двух множеств в одно, и
Find~--- нахождение характерного представителя для множества, содержащего данный
элемент.
Этот представитель может быть использован для проверки принадлежности двух
элементов одному и тому же множеству.
Эта структура данных хорошо изучена и может быть реализована так, что
амортизированная сложность каждой операции~--- всего $O(\alpha(HW))$, где
$\alpha$~--- обратная функция Аккермана.
Для всех практически возможных размеров изображения эта функция не превосходит 5.
}

\texten{
The approach that is named SUF (Scan with Union-Find) in this paper follows.
}\textru{
Далее следует описание подхода, который мы называем SUF (Scan with Union-Find).
}

\texten{
Pixels are processed in the natural order: from top to bottom and from
left to right, so each row is fully processed before all rows after it.
So for each object pixel we encounter there are at most 4 nearby pixels
that were already processed, see \autoref{fig:suf:neighbors}.
}\textru{
Пиксели обрабатываются в естественном порядке сверху вниз слева направо, так что
каждая строка начинает обрабатываться, когда все предыдущие строки уже
обработаны. Таким образом, у очередного пикселя может быть до четырёх пикселей,
которые находятся рядом и уже были обработаны.
}

\texten{
At first each object pixel is considered to belong to its own CC, so right before
considering the neighbors we increment the CC counter.
Moreover, it is associated with its own disjoint set in the Union-Find structure.
More specifically, object pixel at position $x, y$ is associated with
the set number $xW + y$ to guarantee absence of collisions.
}\textru{
Каждый объектный пиксель сначала предполагается принадлежащим своей КС.
Поэтому прежде чем начинать рассматривать соседей, мы увеличиваем счётчик.
Более того того, каждый пиксель изначально ассоциирован со своим собственным
независимым множеством в структуре Union-Find, состоящем из одного элемента.
Конкретно пиксель в позиции $x, y$ заранее ассоциирован с элементом номер $xW + y$,
чтобы гарантировать отсутствие коллизий.
}

\texten{
The possible neighbors are processed in the order shown on \autoref{fig:suf:neighbors},
and for each neighbor that is object pixel, the Union operation is performed
on the two sets containing the current and the neighbor pixel.
In case the Union operation was successful, which means that the pixels
had belonged previously to different sets and we have just found a way to merge two
CC, the CC counter is decremented.
}\textru{
Возможные соседи перебираются в порядке, показанном на
\autoref{fig:suf:neighbors}.
Для каждого объектного пикселя, который напрямую соседствует с текущим, мы
применяем операцию Union для множества, содержащих текущий пиксель и
множества, содержащего этого соседа.
Если эти множества не совпадали, то мы нашли место, где КС соединяются.
В этом случае уменьшаем счётчик на 1.
}

\begin{figure}
  \centering
  \begin{tikzpicture}
    \foreach \x/\y [count=\i from 1] in {0/1,1/1,2/1,0/0} {
      \node[whitebox] at (\x,\y) {\i};
    }
    \node[graybox] at (1, 0) {};
    \node[whitebox] at (1, 0) {};
  \end{tikzpicture}
  \caption{Neighbor mask for SUF algorithm family.}
  \label{fig:suf:neighbors}
\end{figure}

\texten{
Note that this algorithm outputs not the matrix of final labels, but only the number
of CC.
This is the place in which it differs from some common two-pass approaches for CC % TODO references
labeling, as it is an algorithm not for labeling, but just for counting CC.
The latter is exactly what is needed for the aforementioned linear variation problem.
}\textru{
Отметим, что этот алгоритм выдаёт не матрицу финальных меток, а только
количество компонент связности.
Именно в этом месте он отличается от некоторых известных подходов к CC labeling.
Для текущей задачи нам нужно именно считать компоненты связности.
}

\texten{
Also we have to mention that for performance reasons the Union-Find internal structures
are allocated ahead to have the size of at least $HW$ to guarantee that
each object can be easily associated with unique starting set given only its
coordinates. So memory consumption of SUF is linear in the number of pixels.
}\textru{
Также необходимо отметить, что для лучшей производительности служебные массивы
структуры Union-Find заранее выделяются размера как минимум $HW$, чтобы
обеспечить простоту соответствия между пикселями изображения и элементами в
системе множеств.
Таким образом, SUF использует линейное от числа пикселей количество памяти
$O(HW)$.
}

\subsubsection{SUF2}

\texten{
This approach is a modification of SUF designed to reduce memory consumption
and potential number of cache-misses generated by Union-Find operations.
}\textru{
Этот подход~--- модификация SUF, предложенная, чтобы сократить потребление
памяти и предположительно сильно уменьшить число кэш-промахов, происходящих
из-за произвольного доступа к большим массивам структуры Union-Find.
}

\texten{
The idea is to run through each row twice: first time to establish label equivalence
classes within row, and the second time to set unique equal label for all pixels of
same CC.
This gives us the opportunity to maintain just $O(W)$
different labels at each moment, and so the Union-Find structure can be allocated
for only $O(W)$ elements, as opposed to $WH$ elements needed for SUF.
}\textru{
Идея в том, чтобы бежать по каждой строке дважды: один раз чтобы установить
классы эквивалентности меток в текущей и предыдущей строках, и второй раз, чтобы
расставить уникальные метки в пиксели каждого класса.
Это даёт нам возможность иметь не более $O(W)$ различных меток в каждый момент
времени, и поэтому Union-Find можно выделять размера всего $O(W)$, при том что в
SUF выделялось ровно $WH$ элементов.
}

\texten{
The exact maximum number of label equivalence classes between two adjacent rows
depends on whether the given image is condensed.
For non-condensed images $W=2M$, and maximum number of label equivalence classes
in two adjacent rows is $M=\frac{W}2$, as if two object pixels are both in adjacent
rows and in adjacent columns, then they are neighbors in terms of 8-connectivity.
}\textru{
Точная оценка на максимальное число классов эквивалентности меток в двух
соседних строках зависит от того, сжатое изображение поступило на вход, или нет.
Для несжатых изображений $W=2M$, и наибольшее число классов эквивалентности в
соседних строках равно $M=\frac{W}2$, так как если два объектных пикселя
находятся одновременно в соседних строках и соседних столбцах, то они соседи в
смысле 8-связности.
}

\texten{
On the other hand, for condensed images $W=M$, and maximum number of label equivalence
class in two adjacent rows of cells is $2W=2M$.
See \autoref{fig:suf2:extremal} for illustration on the extremal cases.
}\textru{
С другой стороны, для сжатых изображений $W=M$, и максимальное число классов
эквивалентности равно $2W=2M$. Смотри примеры на \autoref{fig:suf2:extremal}.
}

\begin{figure}
  \centering
  \begin{subfigure}{0.475\linewidth}
    \centering
    \begin{tikzpicture}
      \foreach \x in {0,1,...,9} \foreach \y in {0,1} {
        \node[whitebox,dashed,semithick,draw=gray] at (\x,\y) {};
      }
      \foreach \x/\y in {0/0,2/1,4/1,6/0,8/1} {
        \node[whitebox] at (\x,\y) {};
      }
    \end{tikzpicture}
    \caption{Non-condensed image, up to 5 different labels on 10 columns.}
  \end{subfigure}
  \quad
  \begin{subfigure}{0.475\linewidth}
    \centering
    \begin{tikzpicture}
      \foreach \x in {0.5,2.5,...,8.5} \foreach \y in {0.5,2.5} {
        \node[cell] at (\x,\y) {};
      }
      \foreach \x in {0,2,...,8} \foreach \y in {0,3} {
        \node[whitebox] at (\x,\y) {};
      }
    \end{tikzpicture}
    \caption{Condensed image, up to 10 different labels on 5 columns.}
  \end{subfigure}
  \caption{Extremal cases for SUF2.}
  \label{fig:suf2:extremal}
\end{figure}

\texten{
Our benchmarks show that, although SUF2 without condensation is generally more
efficient on noise images than SUF without condensation, the mentioned
consideration make the $2 \times 2$ optimization
of SUF2 less dramatic in means of performance increase factor.
Optimized SUF2 significantly outperforms optimized SUF only on
huge noise images with less than 60\% density, while falling behind on
small and medium-sized noise images.
}\textru{
Проведённые эксперименты показывают, что хотя SUF2 без сжатия в целом более
эффективен, чем SUF без сжатия, приведённая оценка делает относительное
ускорение SUF2 со сжатием не таким значительным.
}

\texten{
Note that, while SUF2 is intended for CC counting, its memory
consumption is only $O(W)$, which is significant for embedded systems and
other resource-tight applications.
}\textru{
Отметим, что SUF2 решает задачу подсчёта СК, используя всего $O(W)$
дополнительной памяти.
}

\subsubsection{DTSUF}

\texten{
This modification of SUF was implemented to compare the optimization we present with
another optimization suggested by Wu, Otoo and Suzuki\cite{wuotoo},
namely decision tree.
Unfortunately DTSUF (Decision Tree Scan plus Union-Find) is not
applicable for condensed image, as it exploits similar observations on 8-connectivity.
}\textru{
Эта модификация SUF была добавлена чтобы сравнить нашу оптимизацию с
одной из оптимизаций, предложенных Wu, Otoo и Suzuki\cite{wuotoo}, а именно с
деревом принятия решения.
К сожалению, DTSUF (Decision Tree Scan with Union-Find) не применим для сжатого
изображения, так как использует свойства 8-связности, которыми наша новая
связность не обладает.
}

\texten{
As before, the goal is to reduce the number of pixel lookup and label copy operations.
This is achieved with careful consideration of only the necessary neighbors,
stopping once the set of one or two label operations that are needed is
determined. The decision tree for processing an object pixel is
shown on \autoref{fig:dtsuf}.
}\textru{
Как и раньше, цель состоит в том, чтобы уменьшить число запросов на содержание
пикселя и копирований меток.
Это достигается аккуратным рассмотрением соседей в оптимальном порядке и
остановкой рассмотрения, как только достаточная для выполнения всех необоходимых
действий информация была получена.
Смотри дерево на \autoref{fig:dtsuf}.
}

\begin{figure}
  \centering
  \begin{subfigure}{0.3\linewidth}
    \centering
    \begin{tikzpicture}
      \node[graybox] at (1,0) {};
      \foreach \x/\y/\a in {0/1/a,1/1/b,2/1/c,0/0/d,1/0/e} {
        \node[whitebox,font=\bf] at (\x,\y) {\a};
      }
    \end{tikzpicture}
    \caption{Neighbor enumeration.}
  \end{subfigure}
  \quad
  \begin{subfigure}{0.65\linewidth}
    \centering
    \begin{tikzpicture}[
      sibling distance=18em,
      lookup/.style = {shape=circle, draw, thick, align=center, font=\bf},
      action/.style = {shape=rectangle, rounded corners, draw, align=center, font=\scriptsize},
      result/.style = {font=\bf}
    ]
      \node[lookup] {b}
        child[level distance=3em,sibling distance=30em] {
          node[lookup] {c}
          child[level distance=5em, sibling distance=14em] {
            node[lookup] {a}
            child[level distance=6em, sibling distance=10em] {
              node[lookup] {d}
              child {
                node[action] {new label}
                edge from parent node[result,above] {$0$}
              }
              child {
                node[action] {union(e,d)}
                edge from parent node[result,above] {$1$}
              }
              edge from parent node[result,above] {$0$}
            }
            child {
              node[action] {union(e,a)}
              edge from parent node[result,above] {$1$}
            }
            edge from parent node[result,above] {$0$}
          }
          child[level distance=9em,sibling distance=24em] {
            node[lookup] {a}
            child[level distance= 6em, sibling distance=10em] {
              node[lookup] {d}
              child {
                node[action] {union(e,c)}
                edge from parent node[result,above] {$0$}
              }
              child {
                node[action] {union(e,c,d)}
                edge from parent node[result,above] {$1$}
              }
              edge from parent node[result,above] {$0$}
            }
            child[sibling distance=20em] {
              node[action] {union(e,c,a)}
              edge from parent node[result,above] {$1$}
            }
            edge from parent node[result,above] {$1$}
          }
          edge from parent node[result,above] {$0$}
        }
        child[level distance=2em] {
          node[action] {union(e,b)}
          edge from parent node[result,above] {$1$}
        };
    \end{tikzpicture}
    \caption{The decision tree.}
  \end{subfigure}
  \caption{DTSUF decision tree optimization.}
  \label{fig:dtsuf}
\end{figure}

\subsection{Condensation types}

For memory-tight applications such as embedded systems it can be useful to
utilise the suggested condensation in a lazy way, so that without actually
constructing the condensed image we use a function (``view'') calculating color of
each cell with 4 lookups to the original binary image.
This is probably most interesting when combined with SUF2 algorithm, as it is a way
to win some time while avoiding memory footprint increase.
However this lazy approach increases the number of image lookups.
% So, according to our benchmark, this combination is efficient only for dense
% ($>30\%$) noise.

So the benchmark charts show three levels of optimization applied to
DFS, SUF and SUF2:
just method name in legend refers to method applied to non-condensed image,
``+view'' means lazy memory-less condensation
and ``+2x2'' means that a temporary condensed image was constructed in memory.
For the latter case, the construction time was included into the overall measured
time of run.

\subsection{Measurement}

For measurement purposes all algorithms and optimizations were implemented by us in \CXX.

The program was compiled using \texttt{g++ 5.3.0-3} compiler with \texttt{-O3} optimization flag.
All tests were run on same machine running under Arch Linux with standard \texttt{linux 4.4.1-2} kernel,
utilising non-overclocked \texttt{Intel(R) Core(TM) i5-2430M CPU @ 2.40GHz} (L3 cache size 3072K) processor and non-overclocked 1333MHz 8GiB Kingston RAM.

The test images were generated in the following manner. At first, 
for given density $d$, the first $4dNM$ pixels of
the $2N \times 2M$ image were filled with object values, while rest filled with background values.
Then all the pixels were permuted using \texttt{std::shuffle} function from \CXX standard library,
which is an implementation of Fisher-Yates Shuffle algorithm\cite{fisher:yates}.

All methods were run on same tests, which was achieved by supplying same seed to the
\texttt{std::mt19937} Mersenne Twister\cite{mt19937} random number engine implementation from \CXX standard library before passing the engine to \texttt{std::shuffle}.

\subsection{Results}

Line charts with benchmarks of mentioned algorithms on different noise images are shown on
\Cref{fig:noise400x600,fig:noise800x1200,fig:noise1200x1800,fig:noise2000x3000,fig:noise4000x6000}.

The extremely bigger CPU time for SUF and DTSUF methods on
huge images could be explained by significant increase in the number of cache-misses
produced by Union-Find.
The charts prove effectiveness of the $2 \times 2$ condensation optimization.  

\newcommand{\inputtable}[2]{
  \begin{figure}
    \centering
    \includegraphics[height=0.45\textheight, keepaspectratio]{pics/#1x#2.eps}
    \caption{Average CPU time per run on noise images $#1 \times #2$ px.}
    \label{fig:noise#1x#2}
  \end{figure}
 % \begin{table}[p]
 %   \centering
 %   \input{tables/#1x#2}
 %   \caption{Average CPU time per run on noise images $#1 \times #2$ px}
 %   \label{tab:noise#1x#2}
 % \end{table}
}

% \inputtable{200}{300}
\inputtable{400}{600}
\inputtable{800}{1200}
\inputtable{1200}{1800}
\inputtable{2000}{3000}
\inputtable{4000}{6000}
% \inputtable{5000}{7500}

% \section{Summary and future work}


\begin{thebibliography}{5}

\bibitem {mak1}
Rudin, L., Osher, S., Fatemi, E.: Nonlinear total variation based noised removal algorithms. Phys. D. 60, 259-268 (1992)

\bibitem {mak2}
Kronrod, A.: On functions of two variables ," Uspehi Mat. Nauk 5, 1(35), 24-134(1950)

\bibitem {mak3}
Makovetskii, A., Kober, V.: Image restoration based on topological properties of functions of two variables.  Proc. SPIE Applications of Digital Image Processing XXXV, 8499, 84990A (2012)

\bibitem {mak4}
4.	Makovetskii, A., Kober, V.: Modified gradient descent method for image restoration. Proc. SPIE Applications of Digital Image Processing XXXVI, 8856, 885608-1 (2013)

\bibitem {mak5}
Chochia, P., Milukova, O.: Comparison of Two-Dimensional Variations in the Context of the Digital Image Complexity Assessment.  Journal of Communications Technology and Electronics, 2015, 60, no. 12,  pp. 1432-1440

\bibitem {hechao}
He, L., Chao, Y., Suzuki, K., Wu, K.:
Fast connected-component labeling. Pattern Recognition. v42, Pages 1977-1987. (2009)

%\bibitem {sterzh}
%Стержанов, М.:
%Методики выделения связных компонент в штриховых бинарных изображениях.
%Труды конференции ГрафиКон’2010. НИУ ИТМО. Санкт-Петербург. с. 169-174. (2010)

\bibitem {CLRS}
Cormen, H., Leiserson, E., Rivest, L., Stein, C.:
Introduction to Algorithms (Second ed.), MIT Press (2001)

\bibitem {wuotoo}
Wu, K, Otoo, E., Suzuki, K.:
Optimizing two-pass connected-component labeling algorithms.
Pattern Anal. Appl. (2008)

\bibitem {fisher:yates}
 Fisher, A., Yates, F.:
 Statistical tables for biological, agricultural and medical research (3rd ed.). 
 London: Oliver \& Boyd. pp. 26–27. OCLC 14222135 (1948)
 
\bibitem{mt19937}
Matsumoto, M., Nishimura, T.:
"Mersenne twister: a 623-dimensionally equidistributed uniform pseudo-random number generator".
ACM Transactions on Modeling and Computer Simulation 8 (1): 3–30. (1998)

\end{thebibliography}

\end{document}

